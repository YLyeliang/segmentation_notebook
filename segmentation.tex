\documentclass{article}

\usepackage[UTF8]{ctex}
\usepackage{amsmath}
\usepackage{graphicx,subfig}
\usepackage{caption}
\usepackage{booktabs}
\usepackage{indentfirst}
\setlength{\parindent}{2em}
\if CLASSOPTIONcompsoc
  \usepackage[nocompress]{cite}
\else
  % normal IEEE
  \usepackage{cite}
\fi



\title{语义分割记录}
\author{叶亮}
\date{\today}
\begin{document} 
\maketitle
\tableofcontents
\section{SF-segnet}
分割任务中影响性能的两个重要因素:分辨率和语义表征能力。常用的方法有:

atrous convolution:在后几个stage使用来保持高分辨率和较强的语义表征. 缺点:较大的计算能力和显存占用。

fpn-like network: 通过双边连接来融合low-level和high-level的特征,提高语义表征能力。

论文核心:fpn-like的连接方式ineffect.提出学习Semantic Flow between layers with different resolutions。即Flow Alignment Module(FAM),通过取相邻level的特征作为输入,输出offset field, and then warp the coarse feature to the fine feature with higher resolution according to the offset field. FAM为即插即用式,可插入到任意backbone中,called \textbf{SFNet}.灵感来源于光流。

在场景解析任务中,主要有两个范式(paradigm)用于高分辨率语义分割。1. 沿主路径keep spatial and semantic information. 2. distributes spatial and semantic information on different parts in a network, then merges back via different strategies.

The first is atrous convolution. The second is fuse multi-level feature maps for both spatiality and semantics

\subsection{Method}
文章采用了Encoder-decoder的架构。其中,Encoder部分为四个阶段的backbone,对应stage1, 2, 3, 4.步长分别为4, 8, 16, 32. Decoder部分为FPN的改进版。将之前的top-down部分的上采样相加模块替换成了Flow Alignment module(FAM). 通过语义流的方式来计算上采样的插值。在pytorch中的实现为:通过torch.grid{\_}sample来实现上采样差值运算。	
\end{document}